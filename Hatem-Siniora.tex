\documentclass[11pt,twoside]{amsart}
\usepackage{amsmath, amsthm, amscd, amsfonts, amssymb, graphicx, color}

\usepackage[utf8]{inputenc}
\usepackage{gensymb}
\usepackage{bm}
\usepackage[most]{tcolorbox}
\usepackage{xcolor}

\usepackage[colorlinks,citecolor=blue]{hyperref}
\usepackage{orcidlink}
\usepackage{cleveref}% http://ctan.org/pkg/cleveref
\addtolength{\topmargin}{-1.5cm}
\linespread {1.3}
\textwidth 17cm 
\textheight 23cm
\addtolength{\hoffset}{-0.3cm}
\oddsidemargin 0cm 
\evensidemargin 0cm
\setcounter{page}{1}
\footskip 0.7cm
%------------------------------------------------------------------------------------% 
\usepackage{fancyhdr}
\pagestyle{fancy}
\fancyhead[LE,RO]{\scriptsize\thepage}
\fancyhead[RE,LO]{\sl\scriptsize Int. J. Group Theory, x no. x (202x) xx-xx \hskip 5 cm Names of authors}
\renewcommand{\headrulewidth}{0pt} % no line in header area
\fancyfoot{} % clear all footer fields
\fancyfoot[CE,CO]{\scriptsize \url{https://dx.doi.org/10.22108/ijgt.---------}} % page number in "outer" position of footer line
\fancyfoot[CE,CO]{\scriptsize \url{https://dx.doi.org/10.22108/ijgt.----------}} 
%------------------------------------------------------------------------------------% 

\DeclareMathOperator{\Aut}{Aut}
\DeclareMathOperator{\GL}{GL}



\newcommand{\N}{\mathbb{N}}
\newcommand{\Z}{\mathbb{Z}}
\newcommand{\Q}{\mathbb{Q}}
\newcommand{\R}{\mathbb{R}}
\newcommand{\F}{\mathbb{F}}

\newtheorem{thm}{Theorem}[section]
\newtheorem{cor}[thm]{Corollary}
\newtheorem{lem}[thm]{Lemma}
\newtheorem{prop}[thm]{Proposition}
\newtheorem{defn}[thm]{Definition}
\newtheorem{rem}[thm]{\bf{Remark}}
\newtheorem{alg}{\bf{Algorithm}}
\numberwithin{equation}{section}
\def\pn{\par\noindent}
\def\cen{\centerline}
%------------------------------------------------------------------------------------%

\begin{document}
%------------------------------------------------------------------------------------%
%%Don not change any thing in this part
\hskip -0.2 cm
\begin{tabular}{c r}
\vspace{-0.6cm}
%\includegraphics[width=1.9cm]{IJGT-logo}\\
\href{https://ijgt.ui.ac.ir/}{\scriptsize  \rm https://ijgt.ui.ac.ir}\\
\end{tabular}
\hfill
\begin{tabular}{l}
\hline
\vspace{-0.2cm}
\scriptsize \rm\bf International Journal of Group Theory\\
\vspace{-0.2cm}
\scriptsize \rm ISSN (print): 2251-7650, ISSN (on-line): 2251-7669 \\
\vspace{-0.2cm}
\scriptsize Vol. {\bf\rm x} No. x {\rm(}202x{\rm)}, pp. xx-xx.\\
\scriptsize $\copyright$ 202x University of Isfahan\\
\hline
\end{tabular}
\hfill
\begin{tabular}{c c}
\vspace{-0.1cm}
%\includegraphics[width=1.9cm]{UI-logo}\\
\href{www.ui.ac.ir}{\scriptsize \rm www.ui.ac.ir}\\
\vspace{-1cm}
\end{tabular}
\vspace{1.3 cm}

%------------------------------------------------------------------------------------%

\title{A Group with Exactly One Noncommutator}
\author{Omar Hatem$^{\orcidlink{0009-0005-3992-8038}}$$^*$}
\author{Daoud Siniora$^{\orcidlink{0000-0003-0929-4087}}$$^*$}

\thanks{{\scriptsize
\hskip -0.4 true cm MSC(2010): Primary: 20F12; Secondary: 20-08.
\newline Keywords: Commutator, Perfect groups, GAP.\\
Communicated by ----------.\\
$^*$ Corresponding author.\\
Received: dd mmmm yyyy, Accepted: dd mmmm yyyy, Published Online: dd mmmm yyyy.\\
\textbf{Cite this article:} Hatem, Siniora, A Group with Exactly One Noncommutator, \emph{Int. J. Group Theory}, \textbf{x} no. x (202x) xx-xx.
{\url{https://dx.doi.org/10.22108/ijgt.------------------}}
}}
\maketitle
%------------------------------------------------------------------------------------%
\begin{abstract} 
The question of whether there exists a finite group of order at least three in which every element except one is a commutator has remained unresolved in group theory. In this article, we address this open problem by developing an algorithmic approach that leverages several group theoretic properties of such groups. Specifically, we utilize Burnside’s theorem for constraining the possible group structures, combined with Plesken and Holt’s extensive enumeration of finite perfect groups, to systematically examine all finite groups up to a certain order for the desired property. The computational core of our work is implemented using the computer system GAP (Groups, Algorithms, and Programming). We discover two nonisomorphic groups of order 368,640 that exhibit the desired property. Our investigation also establishes that this order is the minimum order for such a group to exist. As a result, this study provides a positive answer to Problem 17.76 in the Kourovka Notebook. In addition to the algorithmic framework, this paper provides a structural description of one of the two groups found.
\end{abstract}

%------------------------------------------------------------------------------------%
\bigskip
\bigskip

%------------------------------------------------------------------------------------%

\section{\bf Introduction}

Ore proved that every element in the alternating group $A_n$ is a commutator for $n\geq 5$, and he conjectured the same holds for all nonabelian finite simple groups \cite{Ore1951}. Ore's conjecture was then proven to be true by Liebeck, O'Brien, Shalev, and Tiep \cite{Liebeck2010}. Thus, every element of any finite nonabelian simple group is a commutator. MacHale asked if there exists a finite group of order at least 3 that has exactly one noncommutator element. This is Problem 17.76 of the Kourovka Notebook \cite{khukhro2024unsolvedproblemsgrouptheory}. 

Some progress has been done towards understanding the number of commutators in a finite group. Fite showed that the smallest finite group where the commutator subgroup contains noncommutators is of order 96, see \cite{Fite1902}. Macdonald showed that if $G$ is a group and if $|G:Z(G)|^2 <|G'|$, then there exists elements in $G'$ which are not commutators \cite{Macdonald1986}. Isaacs also has shown that a certain wreath product of an abelian group with a nonabelian group has a commutator subgroup with noncommutators \cite{Isaacs1977}. More examples of groups  where the set of commutators is a proper subset of the commutator subgroup can be found in a survey by Kappe and Morse \cite{Kappe2005}. In this paper, we give a positive answer to MacHale's question. Using the computer algebra system GAP, we found two nonisomorphic groups of order $2^{10}\cdot 360$ which have exactly one noncommutator. Moreover, this is the smallest order of a group with this property. In this paper, we describe the structure of one of them.

We now recall the definitions of the terms used in this paper. Let $G$ be a group and let $g,h\in G$. The \textit{commutator} of $g$ and $h$ is the element $[g,h]=ghg^{-1}h^{-1}$. An element $c\in G$ is a \textit{commutator} if there exist $g\in G$ and $h\in G$ such that $c=[g,h]$. An element of $G$ is a \textit{noncommutator} if it is not a commutator. The \textit{commutator subgroup} $G'$ of $G$ is the subgroup generated by all the commutators of $G$. It follows that the commutator subgroup is the smallest normal subgroup of $G$ such that the quotient of the original group by this subgroup is abelian. A group is called \textit{perfect} if it is equal to its own commutator subgroup. For instance, every nonabelian simple group is perfect.

% While you are preparing your paper, please take care of the following:
% \begin{enumerate}
% \item Dates: Received: 30 April 2009, Accepted: 21 June 2010.\\
% \item MSC2010: Primary only one item; and Secondary at least one item.\\
% \item Key words: At least 3 items and at most 5 items.\\
% \item Authors: Full names, mailing addresses and emails of all authors (if exist).\\
% \item Tags (Formula Numbers): Use \label{A} and \eqref{A}. Remove unused tags.\\
% \item Acknowledgement: At the end of paper but preceding to References.\\
% \item Margins: A long formula should be broken into two or more lines. Empty spaces in the text should be removed.\\
% \item References: Use \cite{ABS} to refer to the specific book/paper [1] in the text. Remove unused references. References should be given in alphabetical order with the following format: a) to books � author, title, publisher, location, year of publication; b) to articles in periodicals or collections � author, title of the article, title of the periodical (collection), volume, year, pagination. Abbreviations of titles of periodicals and collections should be given following Mathematical Reviews at Abbreviations of names of serials, see www.ams.org/msnhtml/serials.pdf\\
% \end{enumerate}

%------------------------------------------------------------------------------------%
\section{Computational Search}
We are searching for a finite group which has exactly one noncommutator. The first such example is cyclic group of order 2. Are there any others? We used the computer algebra system: Groups, Algorithms, Programming (GAP) to perform a search for the desired group. We attempted to first find some key properties about the such a group and its unique noncommutator element that will speed up the search.
\begin{prop}
Let $G$ be a finite group with $|G|\geq 3$ that has exactly one noncommutator element $u\in G$. Then the following hold.  
\begin{itemize}
            \item The order of $u$ is 2.
            \item The element $u$ belongs to the center $Z(G)$ of $G$.
            \item $G$ is a perfect group.
            \item The element $u$ is the product of two commutators. Consequently, the commutator length of $G$ is 2.
\end{itemize}
\label{four_props}
\end{prop}
\begin{proof} Let $u$ be the unique noncommutator element of $G$.
\begin{itemize}
    \item The inverse of a commutator is a commutator, and so the inverse of a noncommutator is a noncommutator. Therefore, the element $u$, being the only noncommutator, must be its own inverse. Thus, $u^2 = 1$.
    \item Since conjugation by an element $g\in G$ is an automorphism of $G$, it follows that $gug^{-1}$ must be a noncommutator as well. Since $u$ is the only noncommutator it must be $gug^{-1}=u$ for any $g\in G$.  Therefore, for any $g\in G$, we have $gu=ug$, which means $u \in Z(G)$.
    \item Let $S$ be the set of all commutators of $G$, so $|S|+1=|G|$. Since $G'=\langle S \rangle\subseteq G$ we get that $|G'|\geq |S| = |G|-1\geq 2$ and either $G'=S$ or $G'=G$. But $|G'|$ divides $|G|$, and thus $|G|=|G'|$ implying that $G=G'$ showing that $G$ is perfect.
    \item Since $|G| > 2$, we can choose some nontrivial commutator element $g\in G$. Then we have that $u=(ug)g^{-1}$. Since $g$ is a commutator, we know $g^{-1}$ is a  commutator as well. If $ug$ were a noncommutator, it follows that $ug=u$ because $u$ is the only noncommutator, and so $g=1$ contradicting that $g$ is nontrivial. So $ug$ is a commutator element too. Therefore, the element $u$ is a product of two commutators, and thus the commutator length of $G$ is 2.
\end{itemize}
\end{proof}

The third point above can be said more generally: Let $G$ be a group of order $n$ with exactly $k$ noncommutators, if $n-k>\frac{n}{2}$, then $G$ is perfect.  Proposition \ref{four_props} tells us that we need to search in the class of finite perfect groups. In GAP we used Plesken and Holt's enumeration of finite perfect groups. We use the properties of Proposition \ref{four_props} together with Burnside's characterization (see below) of commutator elements via character theory to prune the search for greater efficiency. 

\begin{thm}[Burnside \cite{burnsidecomm}]
Let $G$ be a finite group with identity element 1 and let $\operatorname{Irr}(G)$ be the set of all irreducible characters $\chi:G\to \mathbb{C}$ of $G$. Then an element $g\in G$ is a commutator if and only if 
\[
    \sum_{\chi \in \operatorname{Irr}(G)} \frac{\chi(g)}{\chi(1)} \neq 0\,.
    \label{comm_criterion}
\]
\end{thm}

When a group satisfies the properties of Proposition \ref{four_props}, we compute its character table and use Burnside Theorem to count the number of commutators in the group. Our algorithm is presented below. Check the Appendix at the end for the actual GAP code. The reader may also run the entire code in a Juypter Notebook using the following \href{https://github.com/OmarHatemSalem/NonCommutators-GAP}{link}\footnote{https://github.com/OmarHatemSalem/NonCommutators-GAP}. Our algorithm detected one perfect group of order $368640$ which has exactly one noncommutator. The GAP system described this group as the semidirect product \[\mathbb{Z}_2^{10} \rtimes_\varphi A_6\] for some group homomorphism $\varphi:A_6\to \Aut(\Z_2^{10})$. In the next section we give a precise description of the structure of this group.

\begin{tcolorbox}[colback=white, colframe=black!60, title=Pseudocode for the search algorithm]
    \begin{verbatim}
    for group G in perfect groups
            check if commutator length is 2
            check if center is of even order
        if both pass 
            Get CharacterTable tbl of G 
            Get the conjugacy classes of G

        set n = 0
        # n is the number of commutators
                
        #Iterate through every cojugacy class
        for conjugacy class C do:
            set sum = 0
            
            for every irreducible character chi in tbl
                    sum = sum + (chi[C] / chi[1]);
            End loop;

            if sum != 0 then
                    n = n + |C|;
        End loop;
                
            if n+1 = |G| (group found!). 
    Terminate.
    \end{verbatim} \label{alg_for_search}
\end{tcolorbox}



\section{Description of the group $\Z_2^{10} \rtimes_\varphi A_6$}

We first recall the definition of the semidirect product of groups. Given groups $N$ and $H$ together with a group homomorphism $\phi:H\to \Aut(N)$, their semidirect product $N\rtimes_\phi H$ with respect to $\phi$ is the group whose underlying set is the cartesian product $N\times H$ and where for any elements $(n,h)$ and $(n',h')$ in $N\times H$ their product is defined to be
$$(n,h)\,(n',h')=(n\phi_h(n')\,,\,hh')$$
where $\phi_h=\phi(h)$.

In order to describe the group structure of the group $\mathbb{Z}_2^{10} \rtimes_\varphi A_6$ found by the algorithm we need to present its group homomorphism $\varphi:A_6 \to \Aut(\Z_2^{10})$. Recall that the operation of the group $\Z_2=\{0,1\}$ is addition modulo 2, and $A_6$ is the alternating group of degree 6 under function composition.  Observe that $\Z_2^{10}$ is an abelian group of exponent $2$. We now make use of the following fact: Any abelian group $(A,+,0)$ of prime exponent $p$ is a vector space over the finite field $\F_p=\{0,1,\ldots,p-1\}$ with $p$ elements. Here scalar multiplication is defined as follows: for any scalar $m\in\F_p$ and element $a\in A$ we define $m \, a=a+a+\cdots +a ~(m \text{ times})$. Consequently, we may view $\Z_2^{10}$ as a vector space over the field $\F_2$. We think of elements of $\Z_2^{10}$ as column vectors of 0s and 1s of length 10. Moreover, any automorphism of $\Z_2^{10}$ is represented by an invertible $10\times 10$ boolean matrix. Let this be expressed by the group isomorphism $\psi:\Aut(\Z_2^{10})\to \GL_{10}(\mathbb{F}_2)$ which sends an automorphism to the matrix whose columns are the images of the standard basis of $\Z_2^{10}$ under that automorphism. 

Furthermore, to understand the group homomorphism $\varphi:A_6 \to \Aut(\Z_2^{10})$ we only need to know its action on a generating set of $A_6$ to get a complete description of $\varphi$. Given some permutation $\alpha \in A_6$, we let $M_\alpha=(\psi\circ \varphi)(\alpha)$,  this is the invertible $10\times 10$ boolean matrix that represents the automorphism $\varphi(\alpha)\in\Aut(\Z_2^{10})$. The two cycles  $\sigma = (1,2,3,4,5)$ and $\eta = (4,5,6)$ generate $A_6$. Using GAP, we subsequently compute their corresponding matrices $M_\sigma$ and $M_\eta$ given below.
\[   
    M_\sigma = 
    \begin{bmatrix}
    	1 & 1 & 1 & 1 & 1 & 0 & 0 & 0 & 0 & 0 \\
            0 & 0 & 1 & 1 & 1 & 0 & 0 & 0 & 0 & 0 \\
            1 & 0 & 1 & 0 & 1 & 0 & 0 & 0 & 0 & 0 \\
            0 & 1 & 1 & 1 & 1 & 0 & 0 & 0 & 0 & 0 \\
            0 & 1 & 0 & 1 & 1 & 0 & 0 & 0 & 0 & 0 \\
            0 & 0 & 0 & 0 & 0 & 0 & 1 & 0 & 1 & 1 \\
            0 & 0 & 0 & 0 & 0 & 0 & 1 & 0 & 0 & 0 \\
            0 & 0 & 0 & 0 & 0 & 1 & 0 & 1 & 1 & 0 \\
            0 & 0 & 0 & 0 & 0 & 0 & 0 & 1 & 0 & 0 \\
            0 & 0 & 0 & 0 & 0 & 0 & 0 & 0 & 1 & 0 \\
    \end{bmatrix}
     \text{ ~~~ and ~~~ }
    M_\eta = 
    \begin{bmatrix}
    	0 & 1 & 0 & 0 & 1 & 0 & 0 & 0 & 0 & 0 \\
            1 & 1 & 0 & 0 & 1 & 0 & 0 & 0 & 0 & 0 \\
            0 & 1 & 0 & 1 & 1 & 0 & 0 & 0 & 0 & 0 \\
            1 & 0 & 1 & 1 & 0 & 0 & 0 & 0 & 0 & 0 \\
            0 & 0 & 0 & 0 & 1 & 0 & 0 & 0 & 0 & 0 \\
            0 & 0 & 0 & 0 & 0 & 1 & 1 & 0 & 1 & 0 \\
            0 & 0 & 0 & 0 & 0 & 0 & 0 & 0 & 1 & 0 \\
            0 & 0 & 0 & 0 & 0 & 0 & 1 & 0 & 0 & 0 \\
            0 & 0 & 0 & 0 & 0 & 0 & 0 & 1 & 0 & 0 \\
            0 & 0 & 0 & 0 & 0 & 0 & 0 & 0 & 0 & 1 \\
    \end{bmatrix}
    \label{matrices_gens}
\]

With $M_\sigma$ and $M_\eta$ present, and as $\sigma$ and $\eta$ generate $A_6$, we can thus compute $M_\alpha$ for any permutation $\alpha \in A_6$. We can now describe the group operation of $\Z_2^{10} \rtimes_\varphi A_6$. Choose any elements $(s,\alpha)$ and $(t,\beta)$ in $\mathbb{Z}_2^{10} \rtimes_\varphi A_6$. Their product is $$( s,\ \alpha )\,( t,\ \beta ) \ =\ ( s+M_{\alpha } t\ ,\ \alpha \beta ).$$ 
It remains to get $M_\alpha$ to find the product. We can compute $M_\alpha$ using the matrices $M_\sigma$ and $M_\eta$ above. Towards this, we express $\alpha$ as a product of the generators $\sigma$ and $\eta$ of $A_6$ and their inverses. For example, when $\alpha=\sigma^{-2}\eta\sigma^3$, we get

$$M_\alpha=(\psi\circ \varphi)(\alpha)=(\psi\circ \varphi)(\sigma^{-2}\eta\sigma^3)=(M_\sigma)^{-2} M_\eta (M_\sigma)^3.$$

Moreover, the inverse of an element $(s,\alpha)$ in $\mathbb{Z}_2^{10} \rtimes_\varphi A_6$ is $(M_{\alpha^{-1}}s, \alpha^{-1})$. We now give a formula for the commutator in $\mathbb{Z}_2^{10} \rtimes_\varphi A_6$. The commutator is  
\begin{equation}
[( s ,\ \alpha )\ ,\ ( t,\ \beta )]=(s+M_\alpha t+M_{\alpha\beta\alpha^{-1}}s+M_{[\alpha,\beta]}t\ ,\ [\alpha,\beta]).
\end{equation}
This is shown below. 

\begin{align*}
[( s,\ \alpha )\ ,\ ( t,\ \beta )] &= (s,\ \alpha )( t,\ \beta )( s,\ \alpha )^{-1} ( t,\ \beta )^{-1}   \nonumber\\
 &= (s+M_\alpha t\ ,\ \alpha\beta )( M_{\alpha^{-1}}s\ ,\ \alpha ^{-1}) ( M_{\beta^{-1}}t\ ,\ \beta ^{-1})   \nonumber\\
 &= (s+M_\alpha t\ ,\ \alpha\beta )( M_{\alpha^{-1}}s+M_{\alpha^{-1}\beta^{-1}}t\ ,\ \alpha ^{-1}\beta ^{-1})   \nonumber\\
 &= (s+M_\alpha t+M_{\alpha\beta}(M_{\alpha^{-1}}s+M_{\alpha^{-1}\beta^{-1}}t)\ ,\ \alpha\beta \alpha ^{-1}\beta ^{-1})   \nonumber\\
 &= (s+M_\alpha t+M_{\alpha\beta\alpha^{-1}}s+M_{\alpha\beta\alpha^{-1}\beta^{-1}}t\ ,\ \alpha\beta \alpha ^{-1}\beta ^{-1})   \nonumber\\
 &= (s+M_\alpha t+M_{\alpha\beta\alpha^{-1}}s+M_{[\alpha,\beta]}t\ ,\ [\alpha,\beta]).   
\end{align*}


Next, we search for the unique noncommutator element $u$ of $\mathbb{Z}_2^{10} \rtimes_\varphi A_6$ which we know is in the center. Pick any element $(c, \gamma)$ in the center of $\Z_2^{10} \rtimes_\varphi A_6$. Clearly, $\gamma$ must be in the center of $A_6$, which is centerless, and thus, $\gamma$ is the identity permutation $\varepsilon$ of $A_6$. Furthermore, the vector $c$ is an eigenvector of $M_\alpha$ with eigenvalue 1 for all $\alpha \in A_6$. To see this, let $(t,\alpha)$ be any element in $\mathbb{Z}_2^{10} \rtimes_\varepsilon A_6$. Since $(c, \varepsilon)$ is central we get that
\begin{align*}
    (c, \varepsilon)(t, \alpha)&=(t, \alpha)(c, \varepsilon)\\
    (c+M_\varepsilon t\ , \ \varepsilon\alpha)&=(t+M_\alpha c\ , \ \alpha\varepsilon)\\
    (c+t\ , \ \alpha)&=(t+M_\alpha c\ , \ \alpha)
\end{align*}
Thus, $c=M_\alpha\, c$, and so $c$ is an eigenvector of $M_\alpha$ for every $\alpha\in A_6$. Observe that an eigenvector of the generators $M_\sigma$ and $M_\eta$ is an eigenvector of every $M_\alpha$.  This characterization of central elements was used to to get the center of $\mathbb{Z}_2^{10} \rtimes_\varphi A_6$. Using GAP, we compute the intersection of the eigenspaces of $M_\sigma$ and $M_\eta$. We found out that the center of $\Z_2^{10}\rtimes_\varphi A_6$ has 4 elements and one of them is the unique noncommutator $u=(q, \varepsilon)$ where  
\[
    q = 
    \begin{bmatrix}
        1 & 0 & 1 & 0 & 1 & 1 & 1 & 1 & 1 & 1
    \end{bmatrix}.
\]

More information can be found on the group $\mathbb{Z}_2^{10} \rtimes_\varphi A_6$ in \cite{holtplesken}. For a mathematical proof establishing that $u=(q,\varepsilon)$ is the unique noncommutator element of the group $\Z_2^{10}\rtimes_\varphi A_6$, we propose using \cite[Proposition 25]{Serre1977} in Serre's textbook ``Linear Representations of Finite Groups"  where, using the method of ``little groups" of Wigner and Mackey, they describe the irreducible characters of a group $G$ that is the semidirect product of an abelian normal subgroup $A$ with a subgroup $H$. In our setting $A=\Z_2^{10}$ and $H=A_6$.

\bigskip
%------------------------------------------------------------------------------------%

\bibliographystyle{plain}
\bibliography{sample}

% \begin{thebibliography}{20}
% \bibitem{Z}
% E. I. Zel'manov,  On additional laws in the Burnside problem on periodic groups, {\em Internat. J. Algebra Comput.}, {\bf 3} no. 4 (1993) 583--600.

% \bibitem{BEA}
% A. Ballester-Bolinches, R. Esteban-Romero and M. Asaad,  {\em Products of finite groups},
% de Gruyter Expositions in Mathematics, {\bf 53} Walter de Gruyter GmbH \& Co. KG, Berlin, 2010.

% \bibitem{G}
% G. Glauberman, A $p$-group with no normal large abelian subgroup,
%  % {\em Character theory of finite groups}, 61�65, Contemp. Math., 524, Amer. Math. Soc., Providence, RI, 2010.

% \end{thebibliography}

%-----------------------------------------------------------------------------
%-----------------------------------------------------------------------------

\bigskip
\bigskip

{\footnotesize \pn{\bf Omar Hatem}\; \\ {Mathematics and Actuarial Science Department}, {The American University in Cairo, P.O.Box 74,} {Cairo, Egypt}\\
{\tt Email: omarhatem2002@aucegypt.edu}\\

{\footnotesize \pn{\bf Daoud Siniora}\; \\ {Mathematics and Actuarial Science Department}, {The American University in Cairo, P.O.Box 74,} {Cairo, Egypt}\\
{\tt Email: daoud.siniora@aucegypt.edu}\\
{\tt Website \url{https://sites.google.com/view/daoudsiniora}}
\end{document} 